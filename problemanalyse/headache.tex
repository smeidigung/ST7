\section*{Headache as a problem, measured with Fitbit}
\begin{enumerate}
    \item What is headache?
    \item How is headache clinically defined and diagnosed?
    \item How are headaches a biomedical engineering problem?
    \item How does the data from a Fitbit overlap with the symptoms of headache?
    \item Is overlap a necessity for accurate predictions of headaches?
    \item How will we go about collecting data?
\end{enumerate}

\noindent\textbf{What is headache and how is headaches clinically defined and diagnosed}\\
Headaches are grouped as primary and secondary headaches. Primary headaches occurs without an secondary disease and includes migraine, tension headache and cluster headache, whereas secondary headaches has a provable secondary disease. In the primary sector a diagnosis of headaches are done by the general practitioner. Initially a analysis of the patients general condition are done, which includes questions regarding the patients diet, fluid intake and sleep pattern. Secondly, common sign of illness are assessed; skin colour, elevated body temperature etc. If the neither preliminary examinations indicates a second disease, further tests are done in order to rule out primary headaches. These test include assessment of dental state, mobility of the neck, mental state, abnormal speech, abnormal walk, blood pressure etc. \newline

\noindent\textbf{How does the data from a Fitbit overlap with the symptoms of headache?} \\
Some of the data that can be collected from a Fitbit device, is diet and fluid intake. These parameters is known to have influence on headache. Another parameter that also have influence on headaches is sleep patterns. A Fitbit device can be used to track how a person is sleeping by method called actigraphy. Actigraphy is a technology based on the fact
\newline

\noindent\textbf{How are headaches a biomedical engineering problem?} \\
Headaches is a prevalent disorder that affects more than 50\% of the population every year, and more than 90\% have at some point experienced headache. Headache research has been ongoing for decades, and there is no permanent cure or an accurate way to predict the onset of headaches without custom made equipment. A biomedical angle on this may be to provide knowledge of headache patterns and provide help for the people affected. \\

\noindent\textbf{How will we go about collecting data} \\
The collecting of data can be separated in two and the two parts will go on simultaneous. First the user have to fil out a short protocol each time he/her feel headache. The protocol is described in table \ref{tab:headacheProtocol}. The protocol will be used for control the result of the classifier-model. \\
The second part is the collecting of data from Fitbit. The data from Fitbit that will be collected are sleep pattern, heart rate, level of activity and number of steps. One of the reasons for headache are too low fluid intake, therefor the user have to note daily fluid intake with the exception of strong spirits. The study will fokus on the fluid amount and not distinguish between the which kind of fluid it is. The user have to wear the Fitbit watch 24/7. The collecting period for each user isn't decided, because it depends on the number of Fitbit watches there are available. As a minimum collecting of data for each user will be one week. \\

\noindent The collected data will be separated in two groups; training data and test data. The user data will random be divided between the two groups. The size of the groups are approximately equal. Data are classified in two groups, headache and no headache.\\

\noindent\textit{Protocol for collecting headache data:}
\begin{table}[H]
	\makebox[14cm][c]{
\begin{tabular}{m{6.0cm} |m{7.2cm} }
\toprule
\textbf{Name:} & {\textbf{Date:}}   \\ \bottomrule
{When did the headache start?} & {}\\\midrule
{When did the headache stop?} & {}\\ \midrule 
{Grade the severity of the headache}& {Severe / Moderate / Mild}\\\bottomrule
\end{tabular}}
\caption{Protocol for headache}
\label{tab:headacheProtocol}
\end{table}
