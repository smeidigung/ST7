\chapter{Møder og referater}
\section*{Mødeindkaldelse: 12.09.2018}

\subsection*{Mødetidspunkt/-sted:}

Dato: \tabto{7em} 12.09.2018
\newline
Tidspunkt: \tabto{7em} 10:00
\newline
Sted: \tabto{7em} A6-112
\newline
Mødetype: \tabto{7em} Vejledermøde - Opstartsmøde
\vspace{0.5cm}
\hrule
\subsection*{Dagsorden:}
\begin{enumerate}
    \item Forventningsafstemning ift. projekts indhold
    \begin{enumerate}
        \item Hvad er vejledernes formål?
        \item Hvad er vinklen på projetet?
        \item Har i data tilgængeligt?
    \end{enumerate}
    \item Aftale fast ugentligt vejledermøde
    \item Aftale proces indhold
    \begin{enumerate}
        \item Dagsorden/referat
        \item Deadline ift. at fremsende materialer op til et vejledermøde
        \item Litteratursøgning dokumentation
        \item \LaTeX Template skabelon til artikel og poster
    \end{enumerate}
    \item Evt.
\end{enumerate}

\subsection*{Interne noter:}
Hvad er vinklen på projektet?
\begin{enumerate}
    \item Skal vi forudsige forværring/forbedringer i en patients sygdom pba. fitbit data? Tidsaspektet.
    \item Diagnostisering af sygdom hos en patient pba. health data. Model bygget på patienter der har eks. diabetes, kan vi så på data hvor sygdom ikke er kendt diagnostisere diabetes patienter.
\end{enumerate}
Har i data tilgængelig? Vi er ikke interesseret i dataopsamling \\
Sygdom med de største udfald, da det vil gøre det nemmere at se mønstre i data. \\
Opsamle data på en hospitalsafdeling. Sygdom kendt, patient historik er kendt. \\
Hvor har vi data nok? Hvor langt en tidsperiode skal vi opsamle data fra for at det er big data?\\

\subsection*{Referat:}

