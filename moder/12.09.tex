\chapter{Møder og referater}
\section*{Mødeindkaldelse: 12.09.2018}

\subsection*{Mødetidspunkt/-sted:}

Dato: \tabto{7em} 12.09.2018
\newline
Tidspunkt: \tabto{7em} 10:00
\newline
Sted: \tabto{7em} A6-112
\newline
Mødetype: \tabto{7em} Vejledermøde - Opstartsmøde
\vspace{0.5cm}
\hrule
\subsection*{Dagsorden:}
\begin{enumerate}
    \item Forventningsafstemning ift. projekts indhold
    \begin{enumerate}
        \item Hvad er vejledernes formål?
        \item Hvad er vinklen på projetet?
        \item Har i data tilgængeligt?
    \end{enumerate}
    \item Aftale fast ugentligt vejledermøde
    \item Aftale proces indhold
    \begin{enumerate}
        \item Dagsorden/referat
        \item Deadline ift. at fremsende materialer op til et vejledermøde
        \item Litteratursøgning dokumentation
        \item \LaTeX Template skabelon til artikel og poster
    \end{enumerate}
    \item Evt.
\end{enumerate}

\subsection*{Interne noter:}
Hvad er vinklen på projektet?
\begin{enumerate}
    \item Skal vi forudsige forværring/forbedringer i en patients sygdom pba. fitbit data? Tidsaspektet.
    \item Diagnostisering af sygdom hos en patient pba. health data. Model bygget på patienter der har eks. diabetes, kan vi så på data hvor sygdom ikke er kendt diagnostisere diabetes patienter.
\end{enumerate}
Har i data tilgængelig? Vi er ikke interesseret i dataopsamling \\
Sygdom med de største udfald, da det vil gøre det nemmere at se mønstre i data. \\
Opsamle data på en hospitalsafdeling. Sygdom kendt, patient historik er kendt. \\
Hvor har vi data nok? Hvor langt en tidsperiode skal vi opsamle data fra for at det er big data?\\

\subsection*{Referat:}
Fremmødte: \tabto{7em} Jacob, Andreas, Erik, Anne og Ole og Simon \\
\newline
Projekt relateret:
\begin{itemize}
    \item Tele Care Nord har et forsøg kørende hvor de på hjertesvigt patienter opsamler data med Fitbit og søvn apparat placeret under sengen.
    \item Emnet er yderst relevant med et fremtidssyn. På nuværende tidspunkt er det ikke big data, men det bliver det med tiden. Vores projekt starter ved opsamlingspunktet er deres tanke. Men hvor ønsker vi selv at dreje det hen?
    \item Finde correlationer løser ikke i sig selv problemet, men det kan være en forudsætning for at få løst problemet. Problemet løses bedre ved at designe et studie ud fra brugeren og devicet. Får os tættere på virkeligheden.
    \item Begynd med at skrive arbejdsark om problemstillingen. Brug energi på at forstå mønster genkendelses metoderne.
    \item Forslag til cases: Stress vs. ikke-stress, Er der forskel på hverdag og weekend, Sammenligne patienter der er diagnosticeret med stress og personer der ikke er.
    \item Undersøg om det er muligt at finde anonymiseret Fitbit data?
    \item Opsamling af data kunne foregå ved at måle på os selv, som proof of concept.
    \item Fokus skal være på læringsaspektet. 
\end{itemize}
Formalia:
\begin{itemize}
    \item Vejlederrollerne: Ideen er Simons. Primær vejleder er Ole. Simon bliver supplerende vejleder, men belast ham ikke for meget. Mads skal vi ikke inddrage medmindre at Simon og Ole vælger at gøre det.
    \item Ole Vil gerne modtage materiale på skrift fra os inklusiv referater, men feedback gives mundtlig til møderne. Fremsendes senest kl. 12 dagen før et møde. Vi skal gøre opmærksom hvis der er meget som skal læses. I disse situationer skal materialet fremsendes i bedre tid. Ole er meget fleksibel, også ift. planlagte møder.
    \item Vejledermøder: Kig i Ole's kalender og find tidspunkter. Tag helst tid på dage hvor Ole i forvejen har andre møder. Tag ikke de dage hvor kalenderen er helt tom, da Ole formentlig arbejder hjemme disse dage.
    \item Send sms, hvis vi ikke har fået svar på en mail der er vigtig for os.
    \item Arbejdsark er vores dokumentation. Vi gør det for vores egen skyld. Præsenteres pænt og struktureret. Beskriver de mange eksempler, hvorimod artiklen typisk beskriver et scenarie.
    \item Eksamen: Udgangspunkt i artiklen, men vi vil også kunne blive spurgt ind til arbejdsark.
\end{itemize}
\subsection*{Efter mødet}
\textbf{Tidsplan:}
\begin{itemize}
    \item 5  Oktober  - Protokol for dataindsamling starter
    \item 10 Oktober  - Semestergruppemøde
    \item 12 Oktober  - Deadline for protokol for dataindsamling
    \item 7  November - Statusseminar
    \item 22 November - Semestergruppemøde
    \item 30 November - Deadline med udvikling af model
    \item 7  December - Deadline for abstract af artikel
    \item 13 December - Rette periode starter + buffer
    \item 14 December - SEMCON
    \item 20 December - Artikel afleveres.
\end{itemize}
\textbf{Mål og opgaver:} \\
Vi skal vælge et problem at arbejde videre med. Dertil har vi valgt at stress kunne være vores problem, da vi antager det er et udbredt og aktuelt problem og som vi har mulighed for at optage på os selv. Det leder til spørgsmål som; hvordan defineres stress og hvordan måles det normalt, som vi ønsker afklaret. Dette kunne gøres ved at søge littartur om stress.
Dette data skal optages med Fitbit, men hvordan ser data fra Fitbit ud?\\ 
Dertil, hvordan får vi data fra fitbit, som indeholder perioder med og uden stress, og som er markeret/tagged. Heril kunne vi prøve at kombinere fitbit-data fra en forsøgsperson sammen med et spørgeskema om stress. Dette skulle give os mulighed for at markere områder/perioder, før spørgeskemaet er besvaret, som værende "stressende" eller "ikke-stressende".